\documentclass[10pt,a4paper,twoside]{article}
\usepackage[utf8]{inputenc}

\usepackage[top=1cm, bottom=1cm, outer=5.5cm, inner=0.5cm,marginparwidth=3.5cm, marginparsep=1cm]{geometry}

\usepackage{multicol}
\usepackage{titlesec}%Title styling.
\usepackage{graphicx}%Images
\usepackage{paracol}
\usepackage{todonotes}

\usepackage{fbb}%Font package

\usepackage[T1]{fontenc}
\hyphenpenalty=8000
\tolerance=1000

%Vertical styling.
\topmargin = 1cm
\headheight = 0.5cm
\headsep = 0.5cm
\textheight = 25.7cm
\footskip = 1cm
\setlength{\voffset}{-1in}

%Horizontal styling.
\setlength{\columnsep}{1cm}

%Text styling.
\setlength{\parindent}{0em}
\setlength{\parskip}{1.2em}

%Header and footer styling.

%N.B. that even pages are on the left and odd the right.

\begin{document}

HUE.
\clearpage

\title{In Defence of offence and offence}
\author{Joshua Loo}
\date{12 September 2018}

\maketitle

\begin{multicols}{2}

\section{Introduction}

Freedom exists in several different ways. Perhaps the two most important
dimensions of freedom of speech are first, the freedom to make certain
claims without restriction, and second, the ability to make these claims
where and when one wishes. As a reaction to tragedy, some choose to
close themselves off. This is entirely legitimate insofar as it only
affects the individuals who choose to do so. At the point at which it
begins to affect others, it often becomes illegitimate,
counter-productive and coercive itself. This article contains two
defences: first, one of offensive material, not in an intellectual sense
but rather an attack on state or communal attempts at censorship, and
second, a defence of the offence which is taken when such statements are
made, and measures to restrict where and when speech acts take place.

\section{Restrictions on what is
	said}\label{restrictions-on-what-is-said}

Most defences of offence rely in some way on their being a benefit to
truth. Mill offers a three-pronged defence which has become
conventionally cited. This subject has already been analysed in some
depth, both within the school and without. Broadly speaking it is
probably the case that most forms of freedom of speech increase liberal
views because they are right - because we shouldn't regulate what people
do in their beds, we shouldn't jail people for putting things into their
body or other victimless crimes, and we shouldn't force people into
traditional familial institutions if they don't want to take part. It's
also intuitively the case that if one is unprepared to defend a view,
one probably ought not to hold that view in the first place.

More interesting is the status granted to speech - its distinctiveness -
and what this means in relation to offence. Societies (should) prohibit
violence not solely (or, indeed, not at all) because violence is
undesirable, but also because it contains a coercive element which in
turn permits coercive action to be taken against it - it is distinct to
other undesirable actions, like smoking, or poor taste. In its
renunciation of others' autonomy, violence combines with generally
accepted principles of reciprocity, to indicate that its perpetrator
does not deserve full autonomy anymore. In speech, the physical act is
innocuous. The communication of an idea is seen as harmful.

Speech is different to violence in several key respects. First, it is
often a voluntary act to listen to speech. One is rarely forced to
listen to speech. Even when one is in its physical proximity, one does
not have to particularly engage with the syllables entering their ears,
and can ignore what is being said. This contrasts with violence, which,
if not attempted by incompetents, is involuntary. Ergo the idea itself
is not per se impermissible.

Second, it occurs on a different plane to that which violence occurs. It
is often suggested speech itself can be violent; if it is conceptually
violent, the paragraph above provides an adequate response, otherwise it
must contain overt references to violence, which, in themselves, can
also be ignored, but, if delivered as a threat, are not immune to this
criticism. It is at the point that speech contains a credible threat of
violence that it becomes impermissible, because an individual must take
precautions to deal with this possibility. Contrasting a lunatic
complaining about an international conspiracy involving a religious
group with a lunatic threatening to bomb their local place of worship,
the former is indubitably wrong, and one would not say it, but it should
not be punished, just as consenting adults smoking is wrong, but should
not be punished, whereas the latter clearly is wrong, and ought to be
punished, because the credible threat it delivers changes the lives of
the others in an involuntary sense.

Although this somewhat blurs the line between speech and violence, what
it does is it separates speech into the pure communication of an idea at
an intellectual level and its consequences in the physical world. Just
as it is a crime to murder someone by pressing a button, and not a
defence to say, ``I pressed a button - I didn't kill the person'', so
too it ought to be a crime to cause physical involuntary disruption by
saying something, but it is not a crime to simply express an idea. It is
very rare that the expression of an idea's inevitable concomitant is
violence, so at the very least the standard at which speech should be
punished is significantly higher, ie. the speech involved must be
significantly worse, to merit such punishment.

Third, there is a distinction insofar as it is difficult to quantify
acts in this plane. Although our methods of assessing violence are to
some degree cruel, we all have a common frame of reference which is not
entirely arbitrary by which we can assess different levels of violence -
that's why murder is punished more than punching someone without
significantly injuring them. The problem with speech is that it is
difficult to quantify the harm which is claimed to have occurred. This
is because the only discriminant which we can palatably use is whether
or not there has been a complaint. It is illegitimate of one to say that
the harm that a Christian feels when they hear of abortion is better or
worse to that of a victim of a hate crime listening to someone saying
something racist, because we ourselves have not experienced this attack,
and such emotions are fundamentally the products of internalised mental
processes which we cannot understand fully. Telling the Christian that
their experience does not matter is the same as telling someone
suffering from the aftermath of a hate crime that their trauma doesn't
matter or isn't real. This illegitimacy does not exist when we say that,
using objective medical science, one person was less hurt than another.
We therefore have only one binary test for speech acts.

With only one binary test for speech acts, it is the case that we must
punish them all equally. Intuitively, the following is true. It would be
absurd to prohibit all discussion, because, somewhere, the tragedy of
circumstance has most likely caused someone to likely be genuinely
affected by such discussion, whether by being triggered or otherwise.
Similarly, so too would it be to say that we should not speak of the
cruelty of some religious practises because other religious people may
be offended. We therefore cannot also prohibit offensive speech about
sexual minorities or non-binary individuals, because the basis of
offence is that it is arbitrary product of internalised processes.

Importantly, however, this doesn't mean that everything should be
permitted everywhere. In a crowded corridor, most forms of political
speech in the form of leafleting should probably be prohibited, so that
people can move. In a large public square with a history of public
activism, this probably isn't the case. What is shown above is that
these restrictions must be view-agnostic - unlike restrictions on
violence, where we might differentiate between types of violence, we
cannot here, and so we must restrict not by the content of what is said
but how, when and where it is to be said.

\section{Spaces}

Freedom of speech consists not only of what is said but also the
circumstances in which it is said. This is the justification for the
distinction between shouting ``fire'' in a crowded theatre where there
is no fire and shouting ``fire'' in a forest with no one to listen. The
concept of spaces is important, because it means that not every single
space must have the same restrictions or lack thereof, so a refinement
of ideas about free speech is possible.

First, some spaces simply are inappropriate as venues for speech.
Examples in which this is the case should be quite obvious. Primarily,
they are areas which are not only resource-constrained but also areas
through which many people pass and there is no clear alternative. The
voluntary underpinnings of free speech's unique characteristics
evaporate if one is involuntarily subjected to a protest inside a train
station through which one must go. If there's another train station a
hundred metres away, this objection gains less traction. Other spaces
are more obviously appropriate for speech. Again, here, there ought to
still be restrictions. Parliament Square ought to be significantly more
freed up than it was under Blair; that doesn't mean that there should be
no restrictions - a booking system would help to prevent violence,
producing very loud noises might disturb those commuting through
Westminster station, and so on.

Second, spaces should be able to self-define. We can differentiate
between spaces by the sorts of restrictions which can legitimately be
imposed. In a society in which there is huge diversity of views, it's
not clear that governments ought to be able to restrict spaces in ways
other than those view-agnostic. In a group dedicated to a particular
cause, view-agnostic restrictions might be inappropriate. The only
people directly affected by the speech inside a group are its members.
If these members therefore decide to restrict certain views more than
other views, this is an entirely legitimate choice. They are under no
obligation to others' trauma insofar as they are not affected.

Third, spaces which are meant to be intellectually agnostic cannot
justifiably claim that certain views should be restricted because of
their traumatising effect without being under an obligation to ban all
discussion of heated topics. Here, again, there is a distinction between
prohibiting its discussion, and prohibiting its discussion in a public
area where people might have to walk through and listen to it. Given
finite campus police resources, some speakers probably aren't worth
inviting. Given finite time, some probably shouldn't speak anyway
because of the lack of strength in their argument. If members agree
collectively that broad representation is their aim, but this must occur
within the constraints of resource limits, the logic above holds.

Fourth, institutions which are view-agnostic also should not have free
reign over the restrictions that they impose on where, when and how
speech occurs. At some point, free speech is largely meaningless if it
means that one is confined to a ``free speech area'' many kilometres
away from the event that one is protesting against. It also becomes
meaningless if one must hold it at a ridiculous time, restrict the
number of invitees or who listens, or imposes other unreasonable
requirements. The problem is that it's not clear how we can distinguish
between reasonable and unreasonable requirements. Perhaps the most
coherent view is that, given the latent risk of making free speech
meaningless by overly restricting it, there ought to be a presumption in
favour of freedom of speech, and only when a specific exemption is
argued and secured based on previous principles (such as those in the
first part of this essay) can something like this happen.

\section{Safe spaces, trigger warnings and
	no-platforming}\label{safe-spaces-trigger-warnings-and-no-platforming}

With a more coherent framework in which free speech operates, we can ask
ourselves about these innovations in free speech.

First, safe spaces are not an attack on free speech, \emph{per se}.
There is a distinction between unduly restricting discourse within an
open forum, by taking it over, and creating a new space with other
characteristics. The former is questionable, and is often undesirable,
because it turns a space which claims broad representation into a
non-view-agnostic space. This is similar to suggestions of censorship in
the public space. The latter is not an attack in any way. Spaces can
operate concurrently without affecting each other. Even if they do
affect each other, this is often incidental, or voluntary. These
externalities should not affect the fact that freedom of speech and
association imply a freedom to partially define the constitutive
attachments of a group. Even turning formerly non-safe spaces into safe
spaces is not necessarily harmful. Some of these spaces were perhaps
unnecessarily open - without public purpose and yet acting as if they
do.

When the \emph{New York Times} contributing op-ed writer Judith
Shulevitz attacked the idea of safe spaces - \emph{In College and Hiding
	From Scary Ideas} (March 21, 2015), what was missed is that people who
are scared of these ``scary ideas'' aren't going to talk about these
ideas anyway, because they would have simply skipped such discussion.
The conflation isn't necessarily as egregious as it has been in other
cases, but it relies on a slippery slope argument - ``once you designate
some spaces as safe, you imply that the rest are unsafe. It follows that
they should be made safer''. This is false. Life itself is intrinsically
unsafe. The only totally safe world is a world without sentience - if
all humans were dead, none of us could suffer. We all accept a degree of
latent risk, and we accept more risk for more reward. There is no
dichotomy, but a scale. Her argument rests on more egregious free speech
somehow being caused by these innovations - except that only one line is
used to explain why.

Second, attacks on safe spaces themselves intrude on personal liberties.
Safe spaces rely on freedom of speech and association for their basis.
They rely on the harm principle (that acts unharmful to others oughtn't
to be prohibited) to justify the unique restrictions they require. Most
attacks on safe spaces have been principled - they claim that they are,
by their very essence, dangerous. In doing so, they say that the bases
of safe spaces - freedom of speech and association, are dangerous, and
not as absolute as safe spaces would require. This is not a defence of
free speech - it is a perversion of the idea that everyone ought to be
able to say what they wish (unless violence is precipitated), in that it
conflates freedom to say whatever one wishes to say with freedom to say
whenever, however and wherever one wishes.

Third, trigger warnings are clearly not an attack on freedom of speech.
A trigger warning looks like this: ``We are now going to talk about
genocide''. It does not ban discussion of genocide. Indeed, it is
conducive to the very discussion (that about challenging topics) that
trigger warnings' opponents claim to support. Students who do not wish
to talk about traumatic topics probably won't change whether they want
to or not. The question is whether they are actually traumatised. Even
if some of them are making their trauma up, a significant number clearly
aren't. They therefore gain a benefit, ie. the ability to prepare
themselves for a challenging topic, from a costless statement. Those who
benefit from trigger warnings say that often trigger warnings' ability
to prepare them means that they can now talk about topics which they
previously couldn't. Further, specific warning about when a topic is to
be discussed as to a general warning that they will be discussed in a
lecture means that some students turn up to those lectures in the first
place. Even if this only occurs in a small number of circumstances, at
worst, trigger warnings slightly increase discourse about challenging
topics.

Moreover, attacking the permissibility of expressing oneself in this way
is itself an authoritarian imposition. Forcing lecturers to shock rape
victims by springing discussion of rape onto them is not only a
violation of basic human decency - one ought to care about rape victims
- but also a terrible idea from the point of view of freedom of speech,
in that it is forcing people to say things.

Fourth, no-platforming is, very often, legitimate. That is to say that
there is a legitimate debate as to whether or not an institution exists
to promote broad public debate or for the benefit of its members. If an
organisation falls into the latter category it is under no obligation to
invite, or honour an invitation, to any given speaker. If, however, an
organisation expressly exists to facilitate broad debate, the only
legitimate reasons to prevent a speaker are those view-agnostic - they
relate to the finity of resources available. If, as a byproduct, those
speakers who represent a particular ideology are discriminated against
by this method, so be it - this is an unintended consequence, and if an
ideology is intrinsically something which means that those who host it
have to use lots of resources to facilitate an unsound
pseudo-intellectual set of statements, it's probably not an ideology
which ought to be promoted within the free speech paradigm, because it
uses up finite resources. It's obviously the case that a space claiming
to exist for public discussion should not then discriminate against
certain views, but not all institutions claim this as their sole goal.

Some institutions can also have both aims. They can wish to promote the
wellbeing of their members and to promote general discourse. However,
the three differences between speech and violence that have already been
explained means that it is often very difficult to claim that the
wellbeing of their members is in any way significantly impacted in the
first place. If there is a much easier to see claim that freedom of
speech and/or the quality of debate is decreased by not inviting a
speaker, such institutions should prioritise the latter.

Moreover, it's not clear that the distinctions drawn up indicate that
hosting speakers in a view-agnostic way, even in an institution should
be view-agnostic, is something so compulsory that any derogation from
this should be prohibited. What is shown is that drawing such
distinctions is wrong; it is not shown that those who go against this
``should be expelled'' - \emph{Varsity}, \emph{Why Tim Squirrell should
	be expelled} (November 22, 2014). Sometimes institutions slip up - their
imperfections are not grounds to claim that they are entirely imperfect.

Fifth, a space can discriminate against a particular legitimate
characteristic, eg. cogency of argument, and, as a result, because some
ideologies are wrong, or are promoted in the wrong way, appear to be non
view-agnostic. This form of incidental discrimination is still
principally permissible, because it is not discriminating against the
ideology \emph{per se} - it discriminates against another
characteristic. An example might be that racism is wrong, but if one
wants a stunt double the stunt double probably should look similar to
the person who is being copied. Similarly, if a space wishes for a
minimum standard of debate, and, coincidentally, a particular ideology
is unable to meet those minimum requirements as to how speech occurs, it
is permissible that, in removing those who do not use the format
specified, those from one group as opposed to another are removed.

This clarification is important, because this argument should not be
taken as a denial of the very real struggles that those who have
suffered face in dealing with their suffering, or an attack on
self-definition and the creation of teleologically defined spaces.

\section{Changing discussion about free
	speech}\label{changing-discussion-about-free-speech}

The problem with many claims about free speech is that they rely on
misconceptions or confusion. Opposition to safe spaces, trigger warnings
and no-platforming relies on the mistaken equation of different types of
spaces. Once free of the illusion that differentiation between different
traumas is generally possible, many suggestions as to why certain forms
of speech should be banned appear to be less attractive than they would
otherwise. Yet most debate about free speech focuses on these questions.
Without these assumptions, debate can escape its current primitive
level.

A final confusion is this. Freedom of speech fundamentally protects many
incorrect views. A large proportion of views expressed necessarily are
going to be wrong because they are contradictory. That this is so means
that the conflation of criticism and calls for censorship is also
problematic. It enables another form of perverse restriction of debate,
whereby those claiming to ``defend'' free speech actually pervert it.

Both sides of the political divide claim to support free speech. They
claim to maximise free speech with their competing models thereof.
Ultimately, however, if the only defence made of a view is ``free
speech'', this indicates that the strongest thing that can be said for
an argument is that the state should not restrict it. If this is equated
with argument or analysis, truth itself is undervalued.

Truth matters. Even if we cannot find it, it's still our duty to try. In
showing that free speech should be significantly freer than
conventionally suggested, it is also hoped that this is conducive to
determining what is true. Discussion about the boundaries of what is
permissible means that actual discussion - the aim of free speech - is
omitted and ignored. This is not to say that debate about free speech is
itself harmful, but rather that mentioning free speech in debates about
entirely separate issues is unhelpful. Perhaps this is inevitable in a
society in which much of public discourse has already been degraded to
such a high degree, but its effects are so detrimental to the foundation
of the liberal democracy that is crucial to securing our individual
liberties and collective prosperity that we should nevertheless still
resist it.

\end{multicols}

\end{document}