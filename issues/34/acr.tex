The Sixth Form festival is known for producing plays of a very high
standard, taking advantage of talented students both new and old.
\textit{American Classics} continued this trend. Brilliantly entertaining
as well as thought-provoking, all three productions were equally
enjoyable and every cast member performed well. Lightning smooth
transitions and a well-chosen mix of tones make it difficult to find
faults within the play. One criticism of the play as a whole, echoed by
a number of pupils, is its length - at almost three hours long, it was
difficult to comprehend subtler aspects. The three hours, however,
proceeded unflaggingly, without inducing boredom.

The play opened in darkness. Max Raphael narrated each opening scene in
the style of an American circus presenter, with slicked back hair and
the associated voice inflections. We were immediately brought into the
world of American theater. The first scene was one of the opening scenes
from Thorton Wilder's \textit{Our Town} which describes the day to day
lives of the residents of Grover's Corner, a small town in
Massachusetts. However the omniscient narration of this scene is what
provides us with the wider context of the scene, interrupting the
seemingly idyllic town with information about people's deaths, marks the
scene. It was nevertheless the narration in the second part which stood
out. Aryan Mishra's monologue on the eternity of human beings was
beautifully delivered by and acts as in contrast to the first part to
\textit{Our Town}, which marked an ordinary day. The second part showed
one of the characters, recently having died, reminiscing on her past
life, conversing with the dead about the ordinary days that used to
occur. This last scene is extremely powerful, especially considering the
contrast between Mrs. Gibbs in the two scenes, played by Agatha Pethers.
In the first she is cheerful and upbeat, considering selling old
furniture to afford a trip to Paris, which she eventually does for
\$350. In the second part however, she was one of the dead, monotonously
advising Emily Webb on what she should do and not reacting when Dr.
Gibbs places flowers on her grave. This contrast was greatly moving -
coupled with the monologue on human nature, your correspondent found
himself fighting back a single tear, to no avail. As the scene
concluded, the sun rose, ending on a strangely optimistic note full of
ambiguity.

The transitions between each scene were rather striking - smooth and
entertaining, each one very different from the last, ranging from Matt
Carver's shaking of his legs to David Bowie's "Let's Dance" to Theo
Tompkins' hopping surreally through the rest of the cast in slow motion
as a rabbit, they separated each scene and encouraged the audience to
transition from the previous scene.

Unfortunately, one of the few other similarly striking aspects of the
play were the attempts at American accents, and their varying degrees of
success; on the most part they were quite convincing, but there were
times when one was unsure which accent was being attempted, hearing a
strange mix of English and something else that somewhat unidentifiable.
For many, this was not a major issue and did not detract significantly
from their enjoyment of this production.

The play then moved swiftly on, to \textit{Angels in America} by Tony
Kushner, following the life of Prior as he wrestles with HIV in 1980's
New York; in the scene that was chosen, he tells his boyfriend Louis of
his condition. David Edwardes-Ker did an excellent job of conveying the
occasionally hilarious bitterness of Prior, and the almost demented
enthusiasm with which he explains how he is going to die to Louis is
done very well; Louis's (played by Boris Karapetyan) reaction well suits
the audience's similar feelings of discomfort to the news. The second
part, in which a dejected Prior in drag meets a Mormon strung out on
Valium, was probably one of my favorite scenes from the play. Intensely
thought-provoking, both Prior and Harper (played extremely well by
Annika Heath) learn something new from each other in the "threshold of
revelation". Annika gave a hilarious portrayal of Harper and the back
and forth between them was executed very well. The scene ended with a
great monologue by Prior on self identity and worth, and the
realisation, portrayed by David, is suitably emotional and powerful.

Next, in a wonderfully creepy transition involving lots of screaming,
was Arthur Miller's \textit{The Crucible}. The first scene of the two
really captures the panic surrounding the fate of the young girls,
notably Cyci Singh, who showed well the fear that surrounded their
situation, fearing greatly accusations of witchcraft. In a powerful
moment she lashes out against one of her co-conspirators in fear of
being discovered with great anger. Alex Foster as Parris also shows the
fear and nervousness surrounding the affair, fearing that he may be
kicked from his pulpit, with witchcraft having been discovered in his
house. The second part though was very powerful, when Elizabeth Proctor
is arrested for suspicion. Max Raphael as John Proctor stole the show
completely with the immense level of raw emotion he was able to portray
in trying to defend his wife from arrest, and the calm, obliging
response of Rosa Calcraft as Elizabeth was deeply moving. One could hear
a pin drop as he cried out "{[}s{]}he will not go". Yet, the changing of
certain actors for the characters created a little confusion, and it was
hard at times to contextualise events; admittedly, this is a trivial
piece of criticism compared to the rest of the scene.

Next, was the bizarre \textit{Machinal} by Sophie Treadwell, depicting the
life of a nervous, nameless young woman trying to live in the fast pace,
anxiety-inducing world in which she lives. The pace of the first scene
is very well set and quite comedic at times, the repeated phrase "hot
dog" never failed to entertain and one was amazed at the sheer speed at
which the dialogue flew very tightly between the office workers. However
it was the brilliant contrast to this fast pace world, portrayed by
Polly Ruppel, which made the increasingly industrialising world seem
wrong and unnatural; at times the performance was quite chilling.
Especially in the second scene, where she is financially obligated to
marry a man whom she does not love nor feel comfortable around, the
atmosphere is deeply uncomfortable as the scene documents the events of
the honeymoon. The scene became more and more chilling as Jones, whose
self-obsession was wonderfully explored by Leo Doody, puts pressure on
her to do things that she does not want to do; the final piece is her
spiraling into an anxiety attack, heightening the discomfort which had
acted excellently done by Polly.

Tennessee William's extremely entertaining \textit{Glass Menagerie}
followed; it showed Laura hosting a gentleman Caller, Jim, her old high
school crush. Jim's clumsiness and simplicity was perfectly executed by
Theo Tompkins, who by now has mastered the art of the Hugh Laurie-like
glazing of the eyes. The scene unfolds with Laura showing Jim her glass
collection, notably a unicorn which is very precious to her.
Unfortunately during a dance between the two of them, the figure is
knocked off the table, losing its horn. In consoling her Jim kisses her,
regretting his decision, as he is now in a steady relationship. The
painful disappointment of Laura is easily noticeable and well portrayed
by Pandora Mackenzie, in a commentary on identity and individuality.

Next is the pivotal scene from \textit{Who's afraid of Virginia Woolf} in
which George and Martha invite a younger couple back home after a party
and expose their marital problems in a violent way. Beth King was a
hilarious Honey and really made the scene entertaining. Darcy Dixon's
portrayed Martha's taunting manner well, capturing the escalation to
"Violence! Violence!" subtly. Parth Agarwal was also entertaining,
openly hitting on Honey to many laughs from the audience. The anger that
he buildt up in the situation, followed by his sarcastic remark at the
end, was pitch perfect and captured the general hostility well.

Now, the final piece of Act One, Tennessee Williams' \textit{A Streetcar
Named Desire}. In this scene Blanche Dubois, trying to maintain self
respect, as her life falls around her, has to confront her sister's
brutish husband regarding the loss of some property. Rosa Calcraft makes
an excellent Blanche, with one of the better accents in the play, trying
to maintain self dignity, while trying to evade Stanley's anger through
flirting. Stanley, played suitably directly, by Cyrus Gilmartin, is hard
not to root against, as he accuses and insults Blanche. The scene ends
with the happy news of Stella's pregnancy, and the dynamic between the
two sisters is believable and natural. Stella's concern for Blanche was
shown well to the audience by Annika Heath, and we one really empathised
with her throughout.

Finally, starting the Act Two was \textit{The Children's Hour} wherein a
child who is given what she thinks is an unfair punishment takes her
revenge by naming her two headmistresses as lovers, an accusation which
goes on to destroy their careers. Although being very strong, it was one
of the weaker scenes. A few of the accents were hard to pin point and
the general order and dynamic of dialogue were at times a bit shaky.
That being said, this scene was able to create a great tension between
the grandmother, played by Zetta Allas, in a wonderfully hostile manner
and those who stand accused, and the suspense rose throughout the scene
as we in the audience hoped dearly for their innocence to be proved; it
almost is before the final bit of the scene, where Rosalie is coerced
into lying, solidifying Mary's accusation. The heightened frustration of
being falsely accused was portrayed excellently by Viren Shetty, Emily
Hassan and Clara Falkowska as they, outraged, fight back against the
wonderfully dislikeable portrayal of Mary, by Cora Wilson.

In conclusion. \textit{American Classics} was a slick, well executed, well
acted anthology of great American theatre that covered a wide range of
tones and time periods to create an entertaining thought-provoking look
at many themes throughout the show. The transitions were interesting and
very well done. The acting was spot on and at times very moving or very
comedic. The accents, though occasionally imperfect, were on the whole
accurate. It was altogether a great performance, and your correspondent
sincerely enjoyed all three visits.
