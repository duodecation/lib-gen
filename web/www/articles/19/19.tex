\documentclass[10pt,a4paper]{article}
\usepackage[utf8]{inputenc}
\usepackage[margin=1.5cm,includehead,includefoot]{geometry}
\usepackage{enumitem}
\usepackage{fancyhdr}
\usepackage{multicol}

\pagestyle{fancy}
\fancyhf{}
\fancyhead[R]{Editor's note}
\fancyhead[L]{Joshua Loo}
% \fancyfoot[C]{\thepage}

\setlength{\parindent}{0em}
\setlength{\parskip}{1.2em}

\renewcommand{\footrulewidth}{0pt}

\newcounter{count}

\begin{document}
\setcounter{section}{-1}

\title{Editor's note}
\date{4 September 2017}
\author{Joshua Loo}

\maketitle

\begin{multicols}{2}
	
	First, I should like to reiterate everything in the Chair's welcome. The Chair works very hard in his work with the Committee and the library as a whole.
	
	I shall start with a few words on \textit{The Librarian}. We do not have any fixed subject. Rather, we publish anything submitted that we find intellectually stimulating. Articles must meet no requirements, save that they be intellectually stimulating and (negotiably) be less than 10,000 words. Importantly, \textit{The Librarian} \textit{does not represent the views of the Library Committee unless otherwise stated}.
	
	\textit{The Librarian} also publishes \textit{Library News}, a one sheet publication containing library news, games, an agony aunt column and other such content.
	
	This edition of \textit{The Librarian} has been redesigned, as our old readers may notice. This is largely due to inspiration from Felix O'Mahony, who supererogotarily submitted a design for the front page after submissions were sought for a favicon. The font size has also been increased from 10pt to 12pt, after a request by the Committee.
	
	Adventures in Recreational Mathematics has taken an informatic turn, thanks to a stimulating submission from Benedict Randall Shaw. Isky Mathews wrote the past three adventures, and both form the team who publish it.
	
	The cover photo is an image of Ataturk from Wikimedia Commons.
	
	\textit{The Librarian} is now split into three primary sections:
	\begin{enumerate}
		\item Dialectic,
		\item Review, and
		\item Sciences and Mathematics.
	\end{enumerate}
	
	\textit{The Librarian} is typeset in \LaTeX, with Scribus used to create the front page, in 12pt Lato and Computer Modern for mathematics.
	
	The Editor would also like to remind readers that he accepts letters, agony aunt questions and game submissions.
	
	To our old readers, welcome back! To our new readers, welcome!
	
\end{multicols}

\section{Addendum}

\begin{multicols}{2}
	A number of editorial notices have been added to this edition after its initial publication for this web version. They relate to the librarians' clarification to the Editor that attacks on members of the school are not permissible. 
	
	This edition is published for archival and historical purposes, and should not be construed as reflective of an ordinary \textit{Librarian} edition. 
\end{multicols}

\end{document}
